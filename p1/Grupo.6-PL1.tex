\documentclass[a4paper]{article}
\usepackage[utf8]{inputenc}

\title{ Práctica 1 \\
\begin{large}
     Fundamentos de la Ciencia de Datos
\end{large}
}
\author{Samuel Aós Paumard,\\
Enrique Coronado Barco,\\
Carmen Martínez Estévez,\\
Alberto Martínez Ortega}
\usepackage{Sweave}
\begin{document}
\maketitle

\section{Ejericio 1}
\textbf{Realización de un ejercicio en clase con ayuda del profesor en el que se va a realizar un análisis con R de descripción de Datos aplicando todos los conceptos vistos en el tema. Para realizar el ejercicio vamos a utilizar dos ficheros de datos:}
\subsection{}
\textbf{El primer fichero de datos será de tipo .txt, es decir, de texto plano, y estará formado por los datos de los satélites menores de Urano1 que hemos utilizado en la descripción teórica del tema. Lo denominaremos satelites.txt. El objetivo es obtener, utilizando R, los valores de las mismas magnitudes cuyo valor hemos calculado de forma manual.}

Comenzamos con la carga de datos del fichero satelites.txt de la misma forma que vimos durante la clase.
\begin{Schunk}
\begin{Sinput}
> satelites<-read.table("satelites.txt")
\end{Sinput}
\end{Schunk}

Del conjunto de datos extraemos los valores de los radios para su tratamiento, asi como el número de valores que encontramos en esta columna de la tabla de satélites.
\begin{Schunk}
\begin{Sinput}
> Radio<-satelites$Radio
> Radio
\end{Sinput}
\begin{Soutput}
 [1] 13 16 22 33 29 42 27 34 20 30 20 15
\end{Soutput}
\begin{Sinput}
> size_1<-length(Radio)
\end{Sinput}
\end{Schunk}

Calculamos las frecuencias para los valores de los radios.
\begin{Schunk}
\begin{Sinput}
> frec_abs_1<-table(Radio)
> frec_abs_1
\end{Sinput}
\begin{Soutput}
Radio
13 15 16 20 22 27 29 30 33 34 42 
 1  1  1  2  1  1  1  1  1  1  1 
\end{Soutput}
\begin{Sinput}
> frec_abs_acum_1<-cumsum(frec_abs_1)
> frec_abs_acum_1
\end{Sinput}
\begin{Soutput}
13 15 16 20 22 27 29 30 33 34 42 
 1  2  3  5  6  7  8  9 10 11 12 
\end{Soutput}
\begin{Sinput}
> frec_rel_1<-table(Radio)/size_1
> frec_rel_1
\end{Sinput}
\begin{Soutput}
Radio
        13         15         16         20         22         27         29 
0.08333333 0.08333333 0.08333333 0.16666667 0.08333333 0.08333333 0.08333333 
        30         33         34         42 
0.08333333 0.08333333 0.08333333 0.08333333 
\end{Soutput}
\begin{Sinput}
> frec_rel_acum_1<-cumsum(frec_rel_1)
> frec_rel_acum_1
\end{Sinput}
\begin{Soutput}
        13         15         16         20         22         27         29 
0.08333333 0.16666667 0.25000000 0.41666667 0.50000000 0.58333333 0.66666667 
        30         33         34         42 
0.75000000 0.83333333 0.91666667 1.00000000 
\end{Soutput}
\end{Schunk}

Calculamos la media y la mediana, mínimos y máximos de los valores, el rango, la desviación tipica y la varianza.
\begin{Schunk}
\begin{Sinput}
> media_1<-mean(Radio)
> media_1
\end{Sinput}
\begin{Soutput}
[1] 25.08333
\end{Soutput}
\begin{Sinput}
> mediana_1<-median(Radio)
> mediana_1
\end{Sinput}
\begin{Soutput}
[1] 24.5
\end{Soutput}
\begin{Sinput}
> minimo_1<-min(Radio)
> minimo_1
\end{Sinput}
\begin{Soutput}
[1] 13
\end{Soutput}
\begin{Sinput}
> maximo_1<-max(Radio)
> maximo_1
\end{Sinput}
\begin{Soutput}
[1] 42
\end{Soutput}
\begin{Sinput}
> rango_1<-range(Radio)
> rango_1
\end{Sinput}
\begin{Soutput}
[1] 13 42
\end{Soutput}
\begin{Sinput}
> desv_tip_1<-sqrt((sd(Radio)^2)*(size_1-1)/size_1)
> desv_tip_1
\end{Sinput}
\begin{Soutput}
[1] 8.47996
\end{Soutput}
\begin{Sinput}
> varianza_1<-(var(Radio)*(size_1-1))/size_1
> varianza_1
\end{Sinput}
\begin{Soutput}
[1] 71.90972
\end{Soutput}
\end{Schunk}

Para acabar con el tratamiento de los datos calculamos los cuartiles.
\begin{Schunk}
\begin{Sinput}
> cuart1_1<-quantile(Radio,0.25)
> cuart1_1
\end{Sinput}
\begin{Soutput}
25% 
 19 
\end{Soutput}
\begin{Sinput}
> cuart2_1<-quantile(Radio,0.5)
> cuart2_1
\end{Sinput}
\begin{Soutput}
 50% 
24.5 
\end{Soutput}
\begin{Sinput}
> cuart3_1<-quantile(Radio,0.75)
> cuart3_1
\end{Sinput}
\begin{Soutput}
  75% 
30.75 
\end{Soutput}
\begin{Sinput}
> cuart4_1<-quantile(Radio,1)
> cuart4_1
\end{Sinput}
\begin{Soutput}
100% 
  42 
\end{Soutput}
\begin{Sinput}
> cuant54_1<-quantile(Radio,0.54)
> cuant54_1
\end{Sinput}
\begin{Soutput}
 54% 
26.7 
\end{Soutput}
\end{Schunk}

\subsection{}
\textbf{El segundo fichero de datos será de tipo .sav, es decir, un fichero de datos procedente de SPSS. Se denomina cardata.sav y estará formado por datos de automóviles, como su consumo en mpg (millas por galón), cilindrada, aceleración, año de fabricación, modelo, etc.}

%A diferencia del caso anterior el modelo de entrada que tenemos viene dado por un archivo generado en spss (.sav) y, por lo tanto, la función de lectura que tenemos que implementar es distinta a la del ejercicio anterior que leía de archivos de texto. Ahora utilizaremos read_sav proveniente de la libreria \textit{haven}.

\begin{Schunk}
\begin{Sinput}
> install.packages("haven")
> library("haven")      # Librería para la tabla de valores estadísticos
> dataset <- read_sav("cardata.sav")
\end{Sinput}
\end{Schunk}

De este conjunto de datos extraemos los valores de millas por galon (mpg) y, justo despues, para evitar los posibles errores estadísticos provenientes de valores nulos, limpiamos el conjunto de ellos.
\begin{Schunk}
\begin{Sinput}
> mpg<-dataset$mpg
> mpg<-mpg[!is.na(mpg)]
> size_2<-length(mpg)
\end{Sinput}
\end{Schunk}

Tras esto volvemos a repetir las operaciones del ejercicio anterior. \\
Calculamos las frecuencias para los valores de los radios.
\begin{Schunk}
\begin{Sinput}
> frec_abs_2<-table(mpg)
> frec_abs_2
\end{Sinput}
\begin{Soutput}
mpg
15.5 16.2 16.5 16.9   17 17.5 17.6 17.7 18.1 18.2 18.5 18.6 19.1 19.2 19.4 19.8 
   1    1    1    1    2    1    2    1    2    1    1    1    1    3    2    1 
19.9 20.2 20.3 20.5 20.6 20.8 21.1 21.5 21.6   22 22.3 22.4   23 23.2 23.5 23.6 
   1    4    1    2    2    1    1    1    1    1    1    1    2    1    1    1 
23.7 23.8 23.9   24 24.2 24.3   25 25.1 25.4 25.8   26 26.4 26.6 26.8   27 27.2 
   1    1    2    1    1    1    1    1    2    1    1    1    2    1    4    3 
27.4 27.5 27.9   28 28.1 28.4 28.8   29 29.5 29.8 29.9   30 30.4 30.7 30.9   31 
   1    1    1    3    1    1    1    1    1    2    1    2    1    1    1    3 
31.3 31.5 31.6 31.8 31.9   32 32.1 32.2 32.3 32.4 32.7 32.8 32.9   33 33.5 33.7 
   1    1    1    1    1    3    1    1    1    2    1    1    1    1    1    1 
33.8   34 34.1 34.2 34.3 34.4 34.5 34.7   35 35.1 35.7   36 36.1 36.4   37 37.2 
   1    2    2    1    1    1    2    1    1    1    1    5    2    1    3    1 
37.3 37.7   38 38.1   39 39.1 39.4 40.8 40.9 41.5 43.1 43.4   44 44.3 44.6 46.6 
   1    1    4    1    1    1    1    1    1    1    1    1    1    1    1    1 
\end{Soutput}
\begin{Sinput}
> frec_abs_acum_2<-cumsum(frec_abs_2)
> frec_abs_acum_2
\end{Sinput}
\begin{Soutput}
15.5 16.2 16.5 16.9   17 17.5 17.6 17.7 18.1 18.2 18.5 18.6 19.1 19.2 19.4 19.8 
   1    2    3    4    6    7    9   10   12   13   14   15   16   19   21   22 
19.9 20.2 20.3 20.5 20.6 20.8 21.1 21.5 21.6   22 22.3 22.4   23 23.2 23.5 23.6 
  23   27   28   30   32   33   34   35   36   37   38   39   41   42   43   44 
23.7 23.8 23.9   24 24.2 24.3   25 25.1 25.4 25.8   26 26.4 26.6 26.8   27 27.2 
  45   46   48   49   50   51   52   53   55   56   57   58   60   61   65   68 
27.4 27.5 27.9   28 28.1 28.4 28.8   29 29.5 29.8 29.9   30 30.4 30.7 30.9   31 
  69   70   71   74   75   76   77   78   79   81   82   84   85   86   87   90 
31.3 31.5 31.6 31.8 31.9   32 32.1 32.2 32.3 32.4 32.7 32.8 32.9   33 33.5 33.7 
  91   92   93   94   95   98   99  100  101  103  104  105  106  107  108  109 
33.8   34 34.1 34.2 34.3 34.4 34.5 34.7   35 35.1 35.7   36 36.1 36.4   37 37.2 
 110  112  114  115  116  117  119  120  121  122  123  128  130  131  134  135 
37.3 37.7   38 38.1   39 39.1 39.4 40.8 40.9 41.5 43.1 43.4   44 44.3 44.6 46.6 
 136  137  141  142  143  144  145  146  147  148  149  150  151  152  153  154 
\end{Soutput}
\begin{Sinput}
> frec_rel_2<-(table(mpg)/size_2)
> frec_rel_2
\end{Sinput}
\begin{Soutput}
mpg
       15.5        16.2        16.5        16.9          17        17.5 
0.006493506 0.006493506 0.006493506 0.006493506 0.012987013 0.006493506 
       17.6        17.7        18.1        18.2        18.5        18.6 
0.012987013 0.006493506 0.012987013 0.006493506 0.006493506 0.006493506 
       19.1        19.2        19.4        19.8        19.9        20.2 
0.006493506 0.019480519 0.012987013 0.006493506 0.006493506 0.025974026 
       20.3        20.5        20.6        20.8        21.1        21.5 
0.006493506 0.012987013 0.012987013 0.006493506 0.006493506 0.006493506 
       21.6          22        22.3        22.4          23        23.2 
0.006493506 0.006493506 0.006493506 0.006493506 0.012987013 0.006493506 
       23.5        23.6        23.7        23.8        23.9          24 
0.006493506 0.006493506 0.006493506 0.006493506 0.012987013 0.006493506 
       24.2        24.3          25        25.1        25.4        25.8 
0.006493506 0.006493506 0.006493506 0.006493506 0.012987013 0.006493506 
         26        26.4        26.6        26.8          27        27.2 
0.006493506 0.006493506 0.012987013 0.006493506 0.025974026 0.019480519 
       27.4        27.5        27.9          28        28.1        28.4 
0.006493506 0.006493506 0.006493506 0.019480519 0.006493506 0.006493506 
       28.8          29        29.5        29.8        29.9          30 
0.006493506 0.006493506 0.006493506 0.012987013 0.006493506 0.012987013 
       30.4        30.7        30.9          31        31.3        31.5 
0.006493506 0.006493506 0.006493506 0.019480519 0.006493506 0.006493506 
       31.6        31.8        31.9          32        32.1        32.2 
0.006493506 0.006493506 0.006493506 0.019480519 0.006493506 0.006493506 
       32.3        32.4        32.7        32.8        32.9          33 
0.006493506 0.012987013 0.006493506 0.006493506 0.006493506 0.006493506 
       33.5        33.7        33.8          34        34.1        34.2 
0.006493506 0.006493506 0.006493506 0.012987013 0.012987013 0.006493506 
       34.3        34.4        34.5        34.7          35        35.1 
0.006493506 0.006493506 0.012987013 0.006493506 0.006493506 0.006493506 
       35.7          36        36.1        36.4          37        37.2 
0.006493506 0.032467532 0.012987013 0.006493506 0.019480519 0.006493506 
       37.3        37.7          38        38.1          39        39.1 
0.006493506 0.006493506 0.025974026 0.006493506 0.006493506 0.006493506 
       39.4        40.8        40.9        41.5        43.1        43.4 
0.006493506 0.006493506 0.006493506 0.006493506 0.006493506 0.006493506 
         44        44.3        44.6        46.6 
0.006493506 0.006493506 0.006493506 0.006493506 
\end{Soutput}
\begin{Sinput}
> frec_rel_acum_2<-cumsum(frec_rel_2)
> frec_rel_acum_2
\end{Sinput}
\begin{Soutput}
       15.5        16.2        16.5        16.9          17        17.5 
0.006493506 0.012987013 0.019480519 0.025974026 0.038961039 0.045454545 
       17.6        17.7        18.1        18.2        18.5        18.6 
0.058441558 0.064935065 0.077922078 0.084415584 0.090909091 0.097402597 
       19.1        19.2        19.4        19.8        19.9        20.2 
0.103896104 0.123376623 0.136363636 0.142857143 0.149350649 0.175324675 
       20.3        20.5        20.6        20.8        21.1        21.5 
0.181818182 0.194805195 0.207792208 0.214285714 0.220779221 0.227272727 
       21.6          22        22.3        22.4          23        23.2 
0.233766234 0.240259740 0.246753247 0.253246753 0.266233766 0.272727273 
       23.5        23.6        23.7        23.8        23.9          24 
0.279220779 0.285714286 0.292207792 0.298701299 0.311688312 0.318181818 
       24.2        24.3          25        25.1        25.4        25.8 
0.324675325 0.331168831 0.337662338 0.344155844 0.357142857 0.363636364 
         26        26.4        26.6        26.8          27        27.2 
0.370129870 0.376623377 0.389610390 0.396103896 0.422077922 0.441558442 
       27.4        27.5        27.9          28        28.1        28.4 
0.448051948 0.454545455 0.461038961 0.480519481 0.487012987 0.493506494 
       28.8          29        29.5        29.8        29.9          30 
0.500000000 0.506493506 0.512987013 0.525974026 0.532467532 0.545454545 
       30.4        30.7        30.9          31        31.3        31.5 
0.551948052 0.558441558 0.564935065 0.584415584 0.590909091 0.597402597 
       31.6        31.8        31.9          32        32.1        32.2 
0.603896104 0.610389610 0.616883117 0.636363636 0.642857143 0.649350649 
       32.3        32.4        32.7        32.8        32.9          33 
0.655844156 0.668831169 0.675324675 0.681818182 0.688311688 0.694805195 
       33.5        33.7        33.8          34        34.1        34.2 
0.701298701 0.707792208 0.714285714 0.727272727 0.740259740 0.746753247 
       34.3        34.4        34.5        34.7          35        35.1 
0.753246753 0.759740260 0.772727273 0.779220779 0.785714286 0.792207792 
       35.7          36        36.1        36.4          37        37.2 
0.798701299 0.831168831 0.844155844 0.850649351 0.870129870 0.876623377 
       37.3        37.7          38        38.1          39        39.1 
0.883116883 0.889610390 0.915584416 0.922077922 0.928571429 0.935064935 
       39.4        40.8        40.9        41.5        43.1        43.4 
0.941558442 0.948051948 0.954545455 0.961038961 0.967532468 0.974025974 
         44        44.3        44.6        46.6 
0.980519481 0.987012987 0.993506494 1.000000000 
\end{Soutput}
\end{Schunk}

Calculamos la media y la mediana, mínimos y máximos de los valores, el rango, la desviación típica y la varianza.
\begin{Schunk}
\begin{Sinput}
> media_2<-mean(mpg)
> media_2
\end{Sinput}
\begin{Soutput}
[1] 28.79351
\end{Soutput}
\begin{Sinput}
> mediana_2<-median(mpg)
> mediana_2
\end{Sinput}
\begin{Soutput}
[1] 28.9
\end{Soutput}
\begin{Sinput}
> desv_tip_2<-sd(mpg)
> desv_tip_2
\end{Sinput}
\begin{Soutput}
[1] 7.37721
\end{Soutput}
\begin{Sinput}
> var_2<-var(mpg)
> var_2
\end{Sinput}
\begin{Soutput}
[1] 54.42323
\end{Soutput}
\begin{Sinput}
> minimo_2<-min(mpg)
> minimo_2
\end{Sinput}
\begin{Soutput}
[1] 15.5
\end{Soutput}
\begin{Sinput}
> maximo_2<-max(mpg)
> maximo_2
\end{Sinput}
\begin{Soutput}
[1] 46.6
\end{Soutput}
\begin{Sinput}
> rango_2<-range(mpg)
> rango_2
\end{Sinput}
\begin{Soutput}
[1] 15.5 46.6
\end{Soutput}
\end{Schunk}

Para acabar con el tratamiento de los datos calculamos los cuartiles.
\begin{Schunk}
\begin{Sinput}
> cuart1_2<-quantile(mpg,0.25)
> cuart1_2
\end{Sinput}
\begin{Soutput}
  25% 
22.55 
\end{Soutput}
\begin{Sinput}
> cuart2_2<-quantile(mpg,0.5)
> cuart2_2
\end{Sinput}
\begin{Soutput}
 50% 
28.9 
\end{Soutput}
\begin{Sinput}
> cuart3_2<-quantile(mpg,0.75)
> cuart3_2
\end{Sinput}
\begin{Soutput}
   75% 
34.275 
\end{Soutput}
\begin{Sinput}
> cuart4_2<-quantile(mpg,1)
> cuart4_2
\end{Sinput}
\begin{Soutput}
100% 
46.6 
\end{Soutput}
\begin{Sinput}
> cuant54_2<-quantile(mpg,0.54)
> cuant54_2
\end{Sinput}
\begin{Soutput}
54% 
 30 
\end{Soutput}
\end{Schunk}

%---------------------------------------------------------------%
%                             EJERCICIO 2                       %
%---------------------------------------------------------------%

\section{Ejericio 2}
\textbf{Desarrollo por parte de cada grupo del enunciado y la solución de un ejercicio en el que se realice un análisis con R de descripción de Datos introduciendo modificaciones sobre el ejercicio hecho en clase (por ejemplo: los datos se leen desde un fichero generado con Excel o los ficheros que hay en un directorio se listan con la función dir().}

Las mejoras implementadas para esta segunda parte del ejercicio son la carga de datos desde un fichero excel y la presentación de los datos de una forma más ordenada.

Se pretende estudiar, mediante los datos contenidos en un fichero \textit{distacias.xlsx}, las distancias recorridas por una muestra de la población española de las distancias recorridas en sus viajes a lo largo del verano. Con estos datos y junto a un conjunto de nuevas librerías se llevará a cabo el estudio.
Para llevar a cabo las mejoras se requiere de las siguientes librerías:
\begin{itemize}
     \item \texttt{readxl:} para leer del fichero Excel los datos
     \item \texttt{pastecs:} para realizar los cálculos de distintos valores como media y mediana y mínimos y máximos.
     \item \texttt{dplyr:} para trabajar con tablas y poder concatenar columnas con su función mutate.
\end{itemize}

\begin{Schunk}
\begin{Sinput}
> #Instalamos las librerías necesarias
> install.packages("readxl")  #Lectura de fichero Excel
> install.packages("pastecs") #Realiza calculos (media, mediana, ...)
> install.packages("dplyr")   #Mutate - concatencación de columnas
> #Importamos las librerías instaladas previamente
> library("readxl") 
> library("pastecs")
> library("dplyr")
\end{Sinput}
\end{Schunk}

Para podeer leer los datos desde el Excel hemos implementado lo siguiente:
\begin{Schunk}
\begin{Sinput}
> # La función file.choose() abre un explorador de archivos
> # que permite elegir el fichero Excel deseado
> datosExcel<-read_excel(file.choose(),sheet="Hoja1")  
> distancias<-datosExcel$dis
> size_3<-length(distancias)
> distancias
\end{Sinput}
\begin{Soutput}
 [1] 252 288  90 114 460  88 598 501 512 531 539  58  80 432 463 137 461 408 600
[20] 248 511 253 460 217 377 470 429  61 417  90
\end{Soutput}
\end{Schunk}

Realizamos los cálculos de las frecuencias y los cuartiles como hicimos anteriormente
\begin{Schunk}
\begin{Sinput}
> frec_abs_3<-table(distancias)
> frec_abs_acum_3<-cumsum(frec_abs_3)
> frec_rel_3<-table(distancias)/size_3
> frec_rel_acum_3<-cumsum(frec_rel_3)
> cuart1_3<-quantile(distancias,0.25)
> cuart2_3<-quantile(distancias,0.5)
> cuart3_3<-quantile(distancias,0.75)
> cuart4_3<-quantile(distancias,1)
> cuant54_3<-quantile(distancias,0.54)
\end{Sinput}
\end{Schunk}

Para mostrar las frecuencias todas juntas y de una forma más visual hemos creado la siguiente tabla.
\begin{Schunk}
\begin{Sinput}
> frecuencias_3<-data.frame(table(distancias))
> frecuencias_3<-mutate(frecuencias_3,frec_abs_acum_3,frec_rel_3,frec_rel_acum_3)
> frecuencias_3
\end{Sinput}
\begin{Soutput}
   distancias Freq frec_abs_acum_3 frec_rel_3 frec_rel_acum_3
1          58    1               1 0.03333333      0.03333333
2          61    1               2 0.03333333      0.06666667
3          80    1               3 0.03333333      0.10000000
4          88    1               4 0.03333333      0.13333333
5          90    2               6 0.06666667      0.20000000
6         114    1               7 0.03333333      0.23333333
7         137    1               8 0.03333333      0.26666667
8         217    1               9 0.03333333      0.30000000
9         248    1              10 0.03333333      0.33333333
10        252    1              11 0.03333333      0.36666667
11        253    1              12 0.03333333      0.40000000
12        288    1              13 0.03333333      0.43333333
13        377    1              14 0.03333333      0.46666667
14        408    1              15 0.03333333      0.50000000
15        417    1              16 0.03333333      0.53333333
16        429    1              17 0.03333333      0.56666667
17        432    1              18 0.03333333      0.60000000
18        460    2              20 0.06666667      0.66666667
19        461    1              21 0.03333333      0.70000000
20        463    1              22 0.03333333      0.73333333
21        470    1              23 0.03333333      0.76666667
22        501    1              24 0.03333333      0.80000000
23        511    1              25 0.03333333      0.83333333
24        512    1              26 0.03333333      0.86666667
25        531    1              27 0.03333333      0.90000000
26        539    1              28 0.03333333      0.93333333
27        598    1              29 0.03333333      0.96666667
28        600    1              30 0.03333333      1.00000000
\end{Soutput}
\end{Schunk}

A parte hemos creado otra tabla para mostrar la media, la varianza, la desviación típica, el rango, la mediana, el mínimo y el máximo.
\begin{Schunk}
\begin{Sinput}
> tabla_3<-stat.desc(distancias)[c("mean","var","std.dev","range",
+                                    "median","min","max")]
> names(tabla_3)<-c("Media","Varianza","Desviacion tipica","Rango",
+                     "Mediana","Minimo","Maximo")
> tabla_3
\end{Sinput}
\begin{Soutput}
            Media          Varianza Desviacion tipica             Rango 
         338.1667        32621.7989          180.6151          542.0000 
          Mediana            Minimo            Maximo 
         412.5000           58.0000          600.0000 
\end{Soutput}
\end{Schunk}

Por último, para mostrar los cuántiles hemos creado una última tabla.
\begin{Schunk}
\begin{Sinput}
> cuantiles<-c(summary(distancias),cuant54_3)
> cuantiles_3<-cuantiles[order(unlist(cuantiles))]
> cuantiles_3
\end{Sinput}
\begin{Soutput}
    Min.  1st Qu.     Mean   Median      54%  3rd Qu.     Max. 
 58.0000 157.0000 338.1667 412.5000 424.9200 468.2500 600.0000 
\end{Soutput}
\end{Schunk}

\end{document}
